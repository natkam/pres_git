%! Author = natalia
%! Date = 20.02.2020

% Preamble
\documentclass{beamer}

\mode<presentation>
{
\usetheme{Warsaw}
% or Goettingen or Malmoe
\setbeamercovered{transparent}
% or whatever (possibly just delete it)
%\usecolortheme{seahorse}
}

% Packages
\usepackage{amsmath}
\usepackage[english]{babel}
\usepackage[utf8]{inputenc}
\usepackage{graphicx}


\title{Don't Be a Git, Use Git!}
\subtitle{\tiny{(xD)}}
\author{Natalia Maniakowska}
\institute{skygate}
\date{25-02-2020}

\AtBeginSection[]
{
\begin{frame}
    <beamer>{Outline}
    \tableofcontents[currentsection]
\end{frame}
}

\beamerdefaultoverlayspecification{<+->}


% Document
\begin{document}


    \frame{\titlepage}

    \begin{frame}{Outline}
        \tableofcontents
        % You might wish to add the option [pausesections]
    \end{frame}


    %%%%%%%%%%%%%%%%%%%%%%%%%%%%%%%%%%%%%%%%%%%%%%%%
    \section{Introduction}


    \subsection{Definitions}

    \begin{frame}
        \frametitle{What is Git? What it isn't?}
        \begin{itemize}
            \item a \textbf{distributed version control system}
            \item created by Linus Torvalds for development of the Linux cernel (2005!)
            \item free, open source -- GNU General Public License
            \item other DVCS-s: Mercurial, BitKeeper, DCVS\dots (Subversion too, but it's not distributed)
            \item GitHub $\neq$ Git (!)
        \end{itemize}

    \end{frame}

    \begin{frame}
        \frametitle{What is a distributed version control system?}
        \begin{itemize}
            \item \textit{distributed} -- every developer has her/his own copy
            \item \textit{version control} -- keep track of the \textbf{changes}, revert them if needed, etc.
            \item Joel Spolsky (in 2010): \emph{"possibly the biggest advance in software development technology in the [past] ten years"}

            \footnotesize (That was actually about Mercurial, but still\dots)
        \end{itemize}

    \end{frame}


    \begin{frame}
        \frametitle{Git -- definitions}
        \begin{itemize}
            \item \textbf{repository (repo)}
%            \footnotesize{\textbf{repozytorium} \textit{daw}. «szafa lub półka do przechowywania akt urzędowych» }
            \item \textbf{origin}
            \item \textbf{push}
            \item \textbf{pull}
        \end{itemize}

    \end{frame}



    \begin{frame}
        \frametitle{And more definitions\dots}
        \begin{itemize}
            \item \textbf{staging area}
            \item \textbf{commit}
            \item \textbf{branch}
            \item \textbf{master}
            \item \textbf{}
        \end{itemize}

    \end{frame}


%    \bibliography{main}
%    \bibliographystyle{plain}

\end{document}
